\documentclass[runningheads]{llncs}
\usepackage{graphicx}
\usepackage[utf8]{inputenc}
\usepackage{fancyhdr}
\usepackage{hyperref}
\usepackage{listings}
\usepackage{xcolor}
\usepackage{float}

\definecolor{codegreen}{rgb}{0,0.6,0}
\definecolor{codegray}{rgb}{0.5,0.5,0.5}
\definecolor{codepurple}{rgb}{0.58,0,0.82}
\definecolor{backcolour}{rgb}{0.95,0.95,0.92}
\lstdefinestyle{mystyle}{
    backgroundcolor=\color{backcolour},   
    commentstyle=\color{codegreen},
    keywordstyle=\color{magenta},
    numberstyle=\tiny\color{codegray},
    stringstyle=\color{codepurple},
    basicstyle=\ttfamily\footnotesize,
    breakatwhitespace=false,         
    breaklines=true,                 
    captionpos=b,                    
    keepspaces=true,                 
    numbers=left,                    
    numbersep=5pt,                  
    showspaces=false,                
    showstringspaces=false,
    showtabs=false,                  
    tabsize=2
}
\lstset{style=mystyle}
\pagestyle{fancy}
\fancyhf{}
\fancyhf{}
\rhead{ESPE}
\lhead{Universidad de las Fuerzas Armadas}
\rfoot{\thepage}
\renewenvironment{abstract}
{\quotation\small\noindent\rule{\linewidth}{.5pt}\par\smallskip
{\centering\bfseries\abstractname\par}\medskip}
{\par\noindent\rule{\linewidth}{.5pt}\endquotation}
\graphicspath{{../images/}}

\title{UPS Inteligente Utilizando Grafos}
\author{Alexander Guacán\orcidID{L00412289}}
\authorrunning{Universidad de las Fuerzas Armadas}
\institute{Universidad de las Fuerzas Armadas\\
\email{adguacan@espe.edu.ec}}

\begin{document}

    \maketitle

    \begin{abstract}
        % \textbf{Este documento sirve como modelo y guía para su informe. Puede reemplazar los encabezados y el cuerpo del texto con su contenido. Sin embargo, antes de comenzar a escribir, le recomendamos que considere detenidamente las instrucciones de este documento. El resumen describe en palabras concisas lo que hace, por qué lo hace (no necesariamente en este orden) y el resultado principal.}

        % El resumen debe ser autónomo y legible para una persona en el área general. Escriba un breve resumen de su proyecto o producto software, el cual se utiliza para ayudar al lector a determinar rápidamente el propósito de su reporte. Escribir únicamente un párrafo. Se recomienda incluir en el resumen al menos 100 palabras, pero no más de 200.
        
    \end{abstract}

    %%%%%%%%%%%%%%%%%%%%%%%%%%%%%%%%%%%%%%%%%%%%%%%%%%%%%%%%%%%%%%%%%
    %%                        INTRODUCCIÓN                         %%
    %%%%%%%%%%%%%%%%%%%%%%%%%%%%%%%%%%%%%%%%%%%%%%%%%%%%%%%%%%%%%%%%%
    \section{Introducción}
        % La introducción es diferente del resumen; debería elaborar más sobre el contexto del trabajo y otros aspectos. Generalmente, puede repetir algunos de los puntos principales también en la introducción, pero amplíe y use palabras diferentes. Esta sección establece el propósito y los objetivos del escrito. 

        % La introducción generalmente describe el alcance del documento y ofrece una breve explicación del documento. También puede explicar ciertos elementos que son importantes en su proyecto si las explicaciones no forman parte del texto principal. Los lectores pueden tener una idea sobre el siguiente texto antes de empezar a leerlo. Para poder llegar a cumplir con lo antes mencionado, es importante presentar esta introducción de la manera más simple y directa posible. Escriba párrafos concisos y evite hacer listas o enumerar.  El número de párrafos que debe tener la introducción es de 3 a 4, lo cuales deben contener al menos 4 oraciones. 

        % Para evitar el \emph{plagio} y demostrar que ha entendido el tema de estudio, es fundamental parafrasear correctamente las ideas del autor. No copie y pegue partes del artículo, ni siquiera una o dos frases. La mejor manera de hacer esto es dejar el artículo a un lado y escribir su propia comprensión de los puntos clave del autor.

         \href{https://github.com/Alexander-Guacan/Proyecto_Unidad_2.git}{GitHub}.

    %%%%%%%%%%%%%%%%%%%%%%%%%%%%%%%%%%%%%%%%%%%%%%%%%%%%%%%%%%%%%%%%%
    %%                     Desarrollo                              %%
    %%%%%%%%%%%%%%%%%%%%%%%%%%%%%%%%%%%%%%%%%%%%%%%%%%%%%%%%%%%%%%%%%
    \section{Desarrollo}
        % El desarrollo o cuerpo del ensayo en sí está organizado en párrafos. Recuerde que cada párrafo se centra en una idea o aspecto de su tema, y debe contener al menos 4-5 oraciones para que pueda abordar esa idea correctamente.

        % Cada párrafo del cuerpo tiene tres secciones. Primero está la oración principal. Esto le permite al lector saber de qué va a tratar el párrafo y el punto principal que hará. Da el punto del párrafo de inmediato. A continuación, y más grande, están las oraciones de apoyo. Estos amplían la idea central, explicándola con más detalle, explorando lo que significa y, por supuesto, dando la evidencia y el argumento que la respalda. Aquí es donde utiliza su investigación para respaldar su argumento. Luego hay una oración final. Esto reafirma la idea en la oración principal, para recordarle al lector su punto principal~\cite{fest1959}.

        % En la sección cuerpo del ensayo usted puede incluir imágenes las cuales deben estar referencias en el texto de forma mandatoria (vea Figura~\ref{sample_image}).

        

        


    %%%%%%%%%%%%%%%%%%%%%%%%%%%%%%%%%%%%%%%%%%%%%%%%%%%%%%%%%%%%%%%%%
    %%                       DISCUSIÓN                             %%
    %%%%%%%%%%%%%%%%%%%%%%%%%%%%%%%%%%%%%%%%%%%%%%%%%%%%%%%%%%%%%%%%%
    \section{Discusión}
        % Aquí resuma brevemente sus hallazgos. En particular, la sección discusión proporcionará el significado, la importancia y la relevancia de los resultados de su trabajo. Al escribir la discusión del trabajo de investigación, debe concentrarse en proporcionar explicaciones y evaluar los hallazgos. Aquí usted puede incluir los hipervínculos de su video de presentación de su proyecto, así como el de su presentación.

        

    %%%%%%%%%%%%%%%%%%%%%%%%%%%%%%%%%%%%%%%%%%%%%%%%%%%%%%%%%%%%%%%%%
    %%                       CONCLUSIONES                          %%
    %%%%%%%%%%%%%%%%%%%%%%%%%%%%%%%%%%%%%%%%%%%%%%%%%%%%%%%%%%%%%%%%%
    \section{Conclusiones}
        % La sección de conclusión debe reformular su teoría, resumir las ideas de apoyo clave que discutió a lo largo del trabajo y ofrecer su impresión final sobre la idea central. Este resumen final también debe contener la moraleja de su historia o una revelación de una verdad más profunda. Una buena conclusión resumirá sus pensamientos finales y puntos principales, combinando toda la información pertinente con un atractivo emocional para una declaración final que resuene con sus lectores.

    %%%%%%%%%%%%%%%%%%%%%%%%%%%%%%%%%%%%%%%%%%%%%%%%%%%%%%%%%%%%%%%%%
    %%                       BIBLIOGRAFÍA                          %%
    %%%%%%%%%%%%%%%%%%%%%%%%%%%%%%%%%%%%%%%%%%%%%%%%%%%%%%%%%%%%%%%%%
    % \nocite{*}
    % \bibliographystyle{plain}
    % \bibliography{bibliography}

\end{document}

% \lstinputlisting[
%     language=Python,
%     caption=Clase Filter,
%     label=lst:filter_class,
%     firstline=1,
%     lastline=19
% ]{../../Python/src/Filter.py}

% \lstinputlisting[
%     language=Python,
%     caption=Función de filtrado,
%     label=lst:filter_function,
%     firstline=208,
%     lastline=237
% ]{../../Python/src/Shop.py}